\documentclass[10pt,a4paper]{report}
\makeatletter
% commande pour le bas
\def\bas#1#2#3{%
  \hbox to \hsize{%
    \rlap{\vbox{\raggedright #1}}%
    \hss
    \clap{\vbox{\centering #2}}%
    \hss
    \llap{\vbox{\raggedleft #3}}}}%

\def\thickhrulefill{\leavevmode \leaders \hrule height 1pt\hfill \kern \z@}
\renewcommand{\maketitle}{\begin{titlepage}%
    \let\footnotesize\small
    \let\footnoterule\relax
    \parindent \z@
    \reset@font
    \null
    \vskip 50\p@
    \begin{center}
      \hrule
      \vskip 1pt 
      \hrule
      \vskip 1pt
      {\huge \bfseries \strut \@title \strut}\par
      \vskip 1pt
      \hrule
      \vskip 1pt
      \hrule
    \end{center}
    \vskip 50\p@ 
    \begin{center}
      \Large \@author \par
    \end{center}
    \vskip 30\p@
    \begin{center}
      \Large \@date \par
    \end{center}
    \vskip 180 \p@
    \begin{center}
    	\begin{quote}
    		\textit{"Le jeu est la forme la plus \`{e}lev\`{e}e de la recherche."}
    	\end{quote}
    \end{center}
    \begin{flushright}
    Albert Einstein
    \end{flushright}
    \vfil
    \null
  \end{titlepage}%
  \setcounter{footnote}{0}%
}
\makeatother
\usepackage[utf8]{inputenc}
\usepackage[francais]{babel}
\usepackage[T1]{fontenc}
\usepackage{amsmath}
\usepackage{amsfonts}
\usepackage{amssymb}
\usepackage{graphicx}
\author{Jérémi DUPIN, Philippe Escoute}
\title{Othello Ocaml}
\begin{document}
\maketitle
\part{Présentation du projet}
\chapter{Introduction}
\section*{Pourquoi programmer un jeu ?}
Pour deux raisons :
\begin{itemize}
\item Le domaine du jeu est celui qui a le plus apporté à l'intelligence Artificielle. Pour le programmeur, c'est un défi intellectuel qui permet d'améliorer son propre raisonnement.
\item Le problème de jeu est présent dans de nombreux centres de décisions tel que l'économie, la politique, le militaire ...
\end{itemize}

\section*{La théorie des jeux}
Dans le problème des jeux, il y a deux facteurs essentiels : la coopération et la compétition.

\chapter{Le jeu Othello}
\section*{Historique}
Othello a été "inventé" en 1971 par Goro Hasegawa, un japonais. C'est en fait une reprise du jeu Reversi, apparu au 19ème siècle en Angleterre, avec quelques règles supplémentaires et un copyright.
\section*{Règles}

\chapter{Cahier des charges}

\part{Implémentation du jeu}

\part{Intelligence artificielle}

\part{conclusion}

\part{Bibliographie}
\begin{description}
\item[Fédération Française d'Othello] : www.ffothello.org
\end{description}

\end{document}